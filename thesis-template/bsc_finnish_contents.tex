\chapter{Johdanto}


Ohjelmiston arkkitehtuuri on ohjelmiston perusorganisaatio, joka sisältää järjestelmän osat, niiden keskinäiset suhteet ja suhteet ympäristöön \citep{jen_working_2000}. Arkkitehtuurillisiin valintoihin vaikuttaa ainakin käyttökohde, joustavuus, uudelleenkäytettävyys ja ymmärrettävyys \citep{kruchten2004rational}.

Ketterät menetelmät ovat nousseet suosituimmaksi ohjelmistotuotantoprosessiksi \citep{noauthor_14th_2020}. Ketteriä menetelmiä yhdistää iteratiivisesti ja inkrementaalisesti tapahtuva ohjelmiston tuotanto, pienissä erissä tapahtuvat julkaisut, tiivis tiimi sekä ominaisuus- ja tuotebacklogiin perustuva julkaisusuunnitelma \citep{babar_agile_2013}. Ketterille menetelmille olennaista on sopeutuminen vaatimusten muutoksiin \citep{fowler2001agile}. 

Perinteisessä vesiputousmallia noudattavassa ohjelmistotuotannosa kaikki suunnittelu tehtiin tyypillisesti ennen varsinaista sovelluksen toteuttamisvaihetta. Ketterien periaatteiden mukaan tällaista etukäteissuunnittelua tulisi välttää. 

Suurin vastakkainasettelu sijaitsee ketterän kehityksen sopeutumismentaliteetin ja perinteisen suunnittelun ennakoinnin välillä \citep{babar_agile_2013}. Keterän manifestin (\citeyear{fowler2001agile}) mukaan muutokseen reagoimista pidetään tärkeämpänä kuin tarkkojen suunnitelmien noudattamista. Etukäteen tehdyt, mahdollisesti turhiksi osoittautuneet suunnitelmat nähdään turhana työnä. Ihanteena onkin, että päätökset tehdään mahdollisimman myöhäisessä vaiheessa: mitä myöhemmin päätökset tehdään, sitä enemmän tietoa on hyödynnettävissä päätösten tueksi.

Kuitenkin myös ketterissä piireissä on hyväksytty arkkitehtuurisuunnittelun tärkeys laatua parantavana tekijänä: oikealla tavalla suunniteltu arkkitehtuuri mm. vähentää kehitykseen käytettyä aikaa ja kuluja \citep{babar_agile_2013}. Hyvällä arkkitehtuurisuunnittelulla vältytään suurelta määrältä uudelleenohjelmointia kehityksen aikaina \citep{eloranta2015techniques}. Arkkitehtuurisuunnittelun laiminlyöminen voikin johtaa erittäin monimutkaiseen ohjelmistoon: ongelmat voivat johtaa jopa ohjelmiston kehityksen estymiseen \citep{vogel2011software}. 

Eri ketterien menetelmien kuvaukset eivät tyypillisesti kerro mitään arkkitehtuurin suunnittelusta \citep{babar_agile_2013}. Tämän tutkielman tarkoituksena onkin auttaa lukijaa suhtautumaan arkkitehtuurisuunnitteluun oikealla tavalla erilaisissa ketterän sovelluskehityksen konteksteissa.

Luvussa 2 käsitellään apuvälineitä, joilla voidaan vähentää ja helpottaa arkkitehtuurisuunnittelua. Luvussa 3 keskitytään käytänteisiin, miten arkkitehtuurisuunnittelu ketterässä kehityksessä yleisesti toteutetaan. Luvussa 4 pohditaan eri tekijöiden vaikutusta arkkitehtuurisuunnittelun eli millä tavalla eri projektiympäristöissä arkkitehtuurisuunnittelu tulisi suorittaa.

\chapter{Arkkitehtuurin apuvälineet}
Monissa ohjelmistoprojekteissa arkkitehtuurisuunnittelua ei tarvitse tehdä.
Merkittävää arkkitehtuurisuunnittelua esiintyy ainoastaan vähän: suurin osa arkkitehtuuriin liittyvistä päätöksistä on valmiiksi sisällettyinä käytettyihin ohjelmistokehyksiin \citep{bellomo2014agilely}. Elorannan (\citeyear{eloranta2015techniques}) mukaan on esimerkiksi tyypillistä, että mobiiliapplikaatioiden yhteydessä ekosysteemi pakottaa käytetyn arkkitehtuurin ja web-kehityksessä käytetään yleisesti täydellisen ohjelmistoarkkitehtuurin tarjoavia ohjelmistokehyksiä.


\section{Ohjelmistokehykset}
Ohjelmistokehys (software framework) on konkreettinen tai konseptuaalinen alusta, joka toteuttaa yleiskäyttöisiä toiminnallisuuksia, joita voi erikoistaa tai korvata.


\section{Referenssiarkkitehtuurit}
Referenssiarkkitehtuuri on tietylle arkkitehtuurilliselle vaatimukselle syntynyt toimivaksi todettu ratkaisu. Referenssiarkkitehtuurit dokumentoivat järjestelmän rakenteen, järjestelmän rakennuspalaset sekä niiden vastuut ja vuorovaikutukset \citep{vogel2011software}. 

Jos tilanteeseen sopivaa arkkitehtuuria ei ole käytettävissä, on sellainen usein muokattavissa olemassaolevasta referenssiarkkitehtuurista.

\section{Ohjelmarungot}
Kehitettävän sovelluksen pohjaksi voidaan toteuttaa alustava ohjelmarunko.
Ohjelmarungon rakenne vastaa lopullisen järjestelmän arkkitehtuuria, mutta ei vielä toteuta lopullista toiminnallisuutta \citep{vogel2011software}. Ohjelmarunko kehittyy lopulliseksi järjestelmäksi lisäämällä siihen toiminnallisuuksia inkrementaalisesti iteraatioiden aikana.

Cockburn (LÄHDE) esitteleen strategian, jossa projektin alussa rakennetaan ns. kävelevä luuranko (walking skeleton). Kävelevä luuranko on pieni implemaatio järjestelmästä, joka toteuttaa end-to-end -toiminnallisuuden. Tämä ohjelmarunko muodostaa lopullisen arkkitehtuurin sitä mukaa, kun siihen lisätään uusia ominaisuuksia. Kävelevä luuranko voidaan muodostaa esimerkiksi Sprint 0 aikana \citep{eloranta2015techniques}.

Ideana on, että kävelevässä luurangossa kaikki arkkitehtuurin alijärjestelmien yhteydet ovat toteutettuna. Esimerkiksi kerrosarkkitehtuurina toteutetun web-sovelluksen tapauksessa tämä ohjelmarunko voisi toteuttaa jonkin toiminnallisuuden, joka käyttää kaikkea kerrosarkkitehtuurin osia: frontendiä, backendiä ja tietokantaa.


\chapter{Arkkitehtuurisuunnittelun käytänteet}

Arkkitehtuurisuunnittelun voi luokitella ajoituksen mukaan kokonaan etukäteen tehtävään suunnitteluun, osittain etukäteen tehtävään suunnitteluun (Sprint 0) sekä kehityksen aikaiseen suunnitteluun, jossa koko arkkitehtuuri muodostuu täysin inkrementaalisesti. Arkkitehtuurisuunnittelu voi tapahtua myös erillisenä prosessina tiimin ulkopuolella.

Tämän luvun käytänteet vastaa Elorannan väitöskirjassa (2015) tutkittujen ohjelmistokehitystiimien yleisesti käyttämiä käytänteitä.

\section{Koko arkkitehtuurin suunnittelu etukäteen}
Vaikka etukäteissuunnittelu on määritelmällisesti ketterän ideologian vastaista, ketterässä ohjelmistokehityksessä kuitenkin usein harjoitetaan etukäteen kokonaisuudessaan tapahtuvaa arkkitehtuurisuunnittelua \citep{rost_distilling_2015, eloranta2015techniques}. Tässä mallissa arkkitehtuuri suunnitellaan kokonaan ennen siirtymistä muun toiminnallisuuden toteuttamiseen. 

Toteuttamisvaiheessa arkkitehtuuriin tehdään enää korkeintaan pieniä muutoksia ja muutokset tekee arkkitehti, ei ohjelmoija \citep{eloranta2015techniques}. Ongelmana tässä mallissa voidaan pitää, että projektin alkaessa on vain vähän arkkitehtuurillisia päätöksiä tukevaa informaatiota käytettäväksi \citep{waterman_how_2015}.

\section{Sprint 0}

Arkkitehtuurilliset päätökset tulisi tehdä mahdollisimman aikaisin \citep{abrahamsson2010agility}. Mieluiten heti projektin alussa tulisi esimerkiksi päättää, onko järjestelmä hajautettu vai keskitetty, mitä teknologiastackiä käytetään ja niin edelleen \citep{eloranta2015techniques}. Ainakin jotain arkkitehtuurisuunnittelua siis tulisi tehdä heti projektin alussa.

Scrum on tällä hetkellä suosituin ketterän ohjelmistokehityksen viitemalli \citep{noauthor_14th_2020}. Sprintit ovat Scrumin kehitysjaksoja ja ne numeroidaan tyypillisesti yhdestä eteenpäin.

Sprint 0 on tarkoitus alustaa alkava kehitystyö eli hoitaa kaikki projektin aloituksen kannalta oleelliset toimenpiteet ennen varsinaista kehitystyön alkamista. Tässä yhteydessä Sprint 0 tarkoittaa vaihetta, jossa arkkitehtuuri luodaan ensimmäisen iteraation aikana. Tyypillisesti tähän alustavaan arkkitehtuuriin tehdään muutoksia myöhempien sprinttien aikana (ks. Ohjelmistokehykset). 

Sprint 0:lle on tyypillistä, että arkkitehtuurin suunnittelee ohjelmoijat itse, ei erilliset arkkitehdit \citep{eloranta2015techniques}. Sprint 0 on yleensä pituudeltaan yhden normaalin sprintin mittainen, mutta voi kestää myös useamman sprintin \citep{prause_architectural_2012}.

Sprint 0 voidaan kritisoida, koska se ei tuota välitöntä arvoa asiakkaalle.
Scrum guiden mukaan jokaisen sprintin pitäisi tuottaa potentiaalisesti julkaisukelpoinen lisäys tuotteeseen.

\section{Kehityksen aikainen suunnittelu}
Kehityksen aikaisessa suunnittelussa minkäänlaista arkkitehtuurin etukäteissuunnitteluvaihetta ei ole. Arkkitehtuuria suunnitellaan vain tarpeen mukaan järjestelmän ominaisuuksien toteuttamisen yhteydessä. Arkkitehtuuri valmistuu pala palalta kunnes se on valmis.

Ideana on tuottaa alustava arkkitehtuuri ensimmäisen sprintin aikaina samalla kuin toteutetaan tuotteeseen potentiaalisesti tulevia ominaisuuksia \citep{eloranta2015techniques}. Arkkitehtuuria suunnitellaan lähtökohtaisesti vain niitä ominaisuuksia varten, jotka on tarkoitus toteuttaa lähiaikoina \citep{waterman_how_2015}. 

Arkkitehtuurisuunnittelulle voidaan varata aikaa iteraatiosta tai sille voidaan varata koko iteraatio, arkkitehtuurisuunnittelua voi tapahtua muiden suunnitteluaktiviteettien (esim. daily scrum) yhteydessä, arkkitehtuuri voidaan tehdä myös kokonaan ohjelmoinnin yhteydessä \citep{rost_distilling_2015}. 

Yksi toteuttaa arkkitehtuuri sprinttien aikana on määritellä arkkitehtuurisuunnittelu kuten käyttäjätarinat (user storyt) arkkitehtuuritarinoiksi. (Jensen, R. N., Møller, T., Sönder, P. ja Tjørnehøj, developer stories). Scrumissa voidaan myös toteuttaa oma arkkitehtuurisuunnitteluun tarkoitettu sprintti, ns. architecture sprint (Leffingwell). Kehityksen aikainen suunnittelu voi kohdistua arkkitehtuurillisesti tärkeiksi arvioituihin  aspekteihin, jokaiseen user storyyn, jokaiseen epiciin, jokaiseen sprintiin tai koko tuotteeseen \citep{rost_distilling_2015}.

Elorannan (\citeyear{eloranta2015techniques}) mukaan tätä suunnittelukäytäntöä käytti menestyksekkäästi kokeneet tiimit. Kokeneemattomalla tiimillä arkkitehtuurin etukäteissuunnittelun puute johti jatkuvaan refaktorointiin ja lopulta siihen, että arkkitehtuuri piti tehdä uudelleen.


\section{Erillinen arkkitehtuuriprosessi}
Tässä mallissa arkkitehtuurisuunnittelu on eriytetty omaksi prosessikseen. Prosessi tapahtuu tyypilliset täysin erillisessä arkkitehtitiimissä, jossa voi kuitenkin olla samoja jäseniä, kuin itse kehitystiimissä \citep{eloranta2015techniques}.

Arkkitehtuuritiimi ajoittaa arkkitehtuurillisten ominaisuuksien julkaisunsa milestonejen mukaan. Tämä taas määrittää sen, milloin varsinaisen tuotteen toiminnallisia ominaisuuksia voidaan alkaa kehittämään. \citep{eloranta2015techniques}.

Elorannan (\citeyear{eloranta2015techniques}) mukaan syy erilliselle arkkitehtuuriprosessille oli usein, että ei haluttu sotkea arkkitehtuurisuunnittelua Scrum-prosessiin.

\chapter{Arkkitehtuurisuunnittelun määrä}

Vaikka monessa projekteissa, esimerkiksi websovelluksissa, arkkitehtuurisuunnittelua ei tarvitse juurikaan tehdä, joudutaan monessa yhteydessä arkkitehtuuri suunnittelemaan hyvinkin tarkasti. Eloranta (\citeyear{eloranta2015techniques}) mainitsee esimerkkinä monimutkaiset projektit, kuten työkoneiden ohjausjärjestelmät ja lääketieteellisten laitteiden ohjeArkkitehtuuriprototyyppi on toiminnallinen osa järjestelmästä, jolla on tarkoitus saada aikaista palautetta sidosryhmiltä. Prototyyppejä käytetään tyypillisesti suorituskyvyn, muokattavuuden ja rakennettavuuden analysointiin \citep{babar_agile_2013}. 

Spike solutioninka paljon arkkitehtuurisuunnittelua tulisi tehdä etukäteen, vaikuttaa ainakin riski, (odotettavissa oleva) vaatimusten epävakaus, tiimin kulttuuri ja osaaminen sekä myös asiakkaan suhtautuminen ketteryyteen.

\section{Riskin vaikutus ja tunnistaminen}

Arkkitehtuurisuunnittelun määrän tulisi määräytyä sen perusteella, että riski saadaan minimoitua riittävän tyydyttävälle tasolle \citep{fairbanks2010just}. 
\cite{waterman_agility_2018_b} mukaan riskillä on suuri vaikutus siihen, kuinka paljon arkkitehtuurisuunnittelua tulisi tehdä etukäteen: mitä suurempi riski ja sen seuraukset, sitä aikaisemmin siihen pitäisi puuttua ja sitä yksityiskohtaisemmin arkkitehtuuri pitäisi suunnitella. Kyse on siis ketteryyden ja riskin välillä tasapainottelusta.

Mitä enemmän tiimi haluaa vähentää riskiä, sitä aikaisemmin päätökset pitää tehdä. Riskin vähentäminen vaikuttaa negatiivisesti tiimin ketteryyteen \citep{waterman_agility_2018_b}. Jos arkkitehtuurisuunnitteluun käytetään liikaa aikaa, arvon tuotto asiakkaalle viivästyy. Jos suunnittelua ei tapahdu etukäteen, on todennäköisyys epäonnistumiseen suurempi.

Mikä on hyväksyttävä riski vaihtelee paljon: esimerkiksi jos verkkokauppa kaatuu, voidaan menettää asiakkaita, mutta esimerkiksi lääketieteellisessä järjestelmässä voi riskinä olla jopa ihmishengen menettäminen. Tälläisissä tilanteissa on tyypillistä panostaa enemmän etukäteiseen arkkitehtuurisuunnitteluun \citep{waterman_agility_2018_b}. 

\section{Vaatimusten epävakaus}
Vaatimusten epävakaus johtuu yleensä epämääräisesti määritellyistä tai vaihtelevista vaatimuksista \citep{waterman_how_2015}. Epämääräiset vaatimukset johtuvat siitä, ettei asiakas tiedä, mitä haluaa tai heiltä tulee uusia ideoita kehityksen aikana. Vaihtelevat vaatimukset johtuvat siitä, että asiakas muuttaa mieltään tai että käyttötapaukset muuttuvat.

Erityisesti Lean Startup -kontekstissa, jossa periaatteena on luoda tuote asiakkaan käyttäytymisestä tehtävien hypoteesien avulla, järjestelmän vaatimuksista ei ole mitään varmuutta . \citep{reis2011lean}

Mitä enemmän voi olettaa vaatimusten muuttuvan, sitä ketterämpi arkkitehtuurin tulisi olla eli sen pitäisi olla muutosta suvaitseva.

\section{Aikainen arvontuotto}

Jos tarkoituksena on aikainen arvontuotto, täytyy nopeuttaa ensimmäistä julkaisua vähentämällä arkkitehtuurisuunniitteluun käytettyä aikaa \citep{waterman_how_2015}. Tällöin voisi olla järkevää käyttää hyväksi malliarkkitehtuureja sekä inkrementaalista arkkitehtuuria.

\section{Tiimin kulttuurin ja kokemuksen vaikutus}

Elorannan (\citeyear{eloranta2015techniques}) mukaan tiimit, jotka arvostivat ketteriä arvoja valitsivat itse, minkälaista lähestymistapaa arkkitehtuurisuunniitteluun he käyttivät. Tyypillisesti tällöin lähestymistapa oli kehityksen aikainen suunnittelu. Sen sijaan perinteisiä toimintamalleja suosivissa yrityksissä lähestymistapa usein saneltiin johtoportaasta. 

Tiimin arkkitehtuurillinen osaaminen vaikuttaa etukäteissuunnittelun määrään. Mitä parempi osaaminen on, sitä vähemmän suunnitteluun tarvitaan käyttää aikaa. Arkkitehtuurillisesti kokeneella tiimillä on ymmärrys siitä, mikä toimii ja mikä ei \citep{waterman_how_2015}.

Kokenut tiimi voi valita strategiakseen suorittaa arkkitehtuurisuunnittelun kokonaan kehityksen aikana. Elorannan (\citeyear{eloranta2015techniques}) mukaan kokeneet tiimit käyttivät tätä lähestymistapaa menestyksekkäästi.

Osaamisen lisäksi tiimin kyky kommunikoida vaikuttaa siihen, kuinka paljon tarvitaan dokumentaatiota ja etukäteispanostusta ohjelmistokehityksen ohjaamiseksi. Kommunikoinnin kykyyn vaikuttaa kulttuurin lisäksi tiimin koko: mitä suurempi tiimi, sitä enemmän vaaditaan rakennetta ja etukäteissuunnittelua \citep{waterman_how_2015}.

\section{Projektin koko}

Elorannan (2015) mukaan BDUF suunnittelu on tyypillistä varsinkin koodimäärän mukaan mitattuna suurissa projekteissa. 

\section{Asiakkaan vaikutus}
Watermanin (\citeyear{waterman_how_2015}) mukaan asiakkaan ketteryydellä on suuri vaikutus siihen, kuinka paljon etukäteissuunnittelua pitää tehdä eli kuinka ketterää arkkitehtuurisuunnittelu on. Prosessi-orientoitunut asiakas, joka ei usko ketterään ajatusmalliin vähentää huomattavasti tiimin mahdollisuuksia olla ketterä. Asiakas voi haluta hyväksyä kaikki mahdolliset suunnitelmat tai haluaa pakottaa oman prosessimallinsa tiimin sisälle. 

Elorannan (\citeyear{eloranta2015techniques}) tutkimuksessa tuli ilmi, että arkkitehtuurin etukäteissuunnittelua käytettiin eniten projekteissa, joissa kehitettiin sulautettuja järjestelmiä. Elorannan mukaan tämä voi johtua siitä, että laitteistovalmistajat, joiden kanssa yhteistyö tapahtuu, eivät yleensä ole kovin ketteryysmyönteisiä.

Etukäteisbudjettien hyväksymisen tarve voi pakottaa tiimin suunnittelemaan arkkitehtuurin etukäteen, jotta he tietävät, paljonko heillä menee tuotteen valmistamiseen aikaa \citep{waterman_how_2015}. \cite{abrahamsson2010agility} mainitsee mahdollisena ratkaisuna inkrementaalisen rahoitusmallin. 

\chapter{Yhteenveto}

Elorannan (\citeyear{eloranta2015techniques}) mukaan ei ole selvää todistusainoeistoa, jonka mukaan jokin tietty asia vaikuttaisi yksinään siihen, mikä tapa toimia olisi paras.

\chapter{Roskalaatikko}

Kapseloinnin tarkoituksena on, että muutokset vaikuttavat vain mahdollisimman pieneen osaan järjestelmästä. 

Hyvien käytänteiden käyttö ei varsinaisesti vähennä etukäteistyötä, mutta ketteryyden ylläpidettävyyden saavuttamiseksi näitä tulisi kuitenkin käyttää \citep{waterman_agility_2018_a}. 

Ketterällä arkkitehtuurilla tarkoitetaan arkkitehtuuria, joka on suunniteltu ketterää prosessia käyttäen ja on muokattavissa, eli on muutosta suvaitseva \citep{waterman_how_2015}. Ketteränarkkitehtuurin tärkein ominaisuus on muutokseen sopeutuminen. Tämä aikaansaadaan käyttämällä hyväksi hyviä suunnittelukäytänteitä, kuten kapselointia ja selkeää vastuunjakoa, päätösten tekoa viivyttelemällä sekä suunnittelemalla arkkitehtuuri niin, että vaihtoehdoille jätetään tilaa \citep{waterman_agility_2018_a}. 

Hyvistä suunnittelu/koodauskäytänteistä (Xp jne)

Päätösten viivyttelystä

Ketterä arkkitehtuuri tulisi suunnitella niin, että tulevaisuudessa eteen tulevat muutokset ja uudet vaatimukset ovat mahdollista sisällyttää sovellukseen. Tässä auttaa, kun tiedostaa asiat, joita saatetaan joutua muuttamaan myöhemmin sekä se, että vältetään arkkitehtuurin liiallista ennenaikaista optimointia jotain tiettyä käyttötarkoitusta varten 
\citep{waterman_agility_2018_a}.  

Arkkitehtuuriprototyyppi on toiminnallinen osa järjestelmästä, jolla on tarkoitus saada aikaista palautetta sidosryhmiltä. Prototyyppejä käytetään tyypillisesti suorituskyvyn, muokattavuuden ja rakennettavuuden analysointiin \citep{babar_agile_2013}. 

Spike solution